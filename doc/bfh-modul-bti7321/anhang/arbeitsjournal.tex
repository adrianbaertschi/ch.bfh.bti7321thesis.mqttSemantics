\chapter{Arbeitsjournal}
\label{chap:arbeitsjournal}

\begin{table}[h!]
\begin{tabularx}{\textwidth}{|l|l|X|}

 \hline
 {\bf Woche } & {\bf Datum }  & {\bf Arbeiten }  \\ 
 \hline
  1  & 14.09. &  Kickoff-Vorlesung BFH Bernhard Anrig \newline Projektinitialisierung, Aufsetzen Dokumentation auf sharelatex.com \newline Besprechung mit Reto König betreffend Ablauf, Organisatorisches, Ziele  \\ \hline
 
  2  & 21.09. &  Projektplanung \newline Dokumentation Projektmanagement, Dokumentation Einleitung    \\ \hline
  3  & 28.09. &  Recherche bestehende Konzepte \newline Einarbeitung Tinkerforge Bausteine \newline Aufsetzen Prototyp Tinkerforge   \\ \hline
  
  4  & 05.10. &  Einlesen bestehende Technologien (Coap, LWM2M, RESTful API Dokumentation) \newline Besprechung mit Reto König, skizzieren des Prototyps     \\ \hline
  
  5  & 12.10. &  Prototyp Tinkerforge Temperatursensor mit automatischer Enumeration/Erkennung, Versenden der Werte per MQTT \newline Einbindung MQTT Topic Tree Webapp für bessere Übersicht     \\ \hline
  
  6  & 19.10. &  Übernahme der Prototyp Ergenisse in neues Projekt, Refactoring  \newline Server Setup DigitalOcean. Installation Docker, Mosquitto Broker und Apache Webserver \newline Analyse IBM IoT Foundation      \\ \hline
  
  7  & 26.10. &  Einbinden von Tinkerforge DualButton und Joystick \newline Probleme mit Online Dokumentation auf sharelatex.com, Aufsetzen und Einrichtung lokale Latex Umgebung \newline Umstrukturierung der Applikation \newline Erzeugung Device Description in JSON \newline Einführung einheitliches Logging     \\ \hline
  
  8  & 02.11. &  Besprechung mit Federico Flueckiger (Einführunhg in Thema, Termin Verteidigung)    \newline  Ansatz für generische Einbindung der Tinkerforge Bricklets mittels Reflection des Java APIs. \newline Besprechung mit Reto König betreffend Termin Verteitigung, Poster, Book   \\ \hline
  
  9  & 09.11. &  Generischer Reflection Ansatz verworfen, nicht praxistauglich  \newline Organisation Termin für Verteidigung \newline Verfeinerung Device Description   \\ \hline
  
 10  & 16.11. &  Dokumentation Konzept, Architektur \newline Besprechung mit Reto König betreffend TODO  \\ \hline
 
 11  & 23.11. &  Initialisierung Webapplikation für Device Description Anzeige    \\ \hline
 
 12  & 30.11. &  Besprechung mit Reto König betreffend TODO   \\ \hline
 
 13  & 07.12. &  Commands   \\ \hline
 
 14  & 14.12. & Poster für Finalday \newline BFH Book Page   \\ \hline
 
 15  & 11.01. & Trennung Library und Tinkerforge-Demo \newline Git Repo Reorg  \\ \hline
 
 16  & 18.01. & Commands  \\ \hline
 
\end{tabularx}
\caption{Arbeitsjournal}
\end{table}