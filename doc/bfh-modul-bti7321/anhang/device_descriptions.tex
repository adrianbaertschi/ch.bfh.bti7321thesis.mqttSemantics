\chapter{Device Descriptions Tinkerforge}
\label{app:device_descriptions}

\textbf{Humidity Bricklet}

\begin{listing}[H]
\begin{minted}[frame=single,
               framesep=3mm,
               linenos=false,
               xleftmargin=21pt,
               tabsize=4,
               fontsize=\footnotesize]{yaml}

---
id: "IoT-Humidity Bricklet"
version: "0.0.1"
description: "The Humidity Bricklet can be used to measure relative humidity. 
  \ The measured humidity can be read out directly in percent, no conversions 
  \ are necessary, with configurable interval"
stateDescription:
  states:
  - name: "HumidityInterval"
    range:
      min: 0
      max: 9223372036854775807
      type: "Long"
    description: "Interval of the measurements in ms."
eventDescription:
  events:
  - name: "Humidity"
    range:
      min: 0.0
      max: 100.0
      type: "Double"
    description: "Relative Humidity in percent"
commandDescription:
  commands:
  - name: "SetInterval"
    linkedState: "HumidityInterval"
    description: "Set the measurement interval, 0 disables the measurements"
    parameter:
      Interval:
        min: 0
        max: 9223372036854775807
        type: "Long"
complexTypes: []

\end{minted}
\caption{YAML Device Description Humidity Bricklet}
\end{listing}




\textbf{Temperatur IR Bricklet}

\begin{minted}[frame=single,
               framesep=3mm,
               linenos=false,
               xleftmargin=21pt,
               tabsize=4,
               fontsize=\footnotesize]{yaml}

---
id: "IoT - Temperature IR Bricklet"
version: "0.0.1"
stateDescription:
  states:
  - name: "AmbientTemperatureInterval"
    range:
      min: 0
      max: 9223372036854775807
      type: "Long"
    description: "Time in ms"
  - name: "ObjectTemperatureInterval"
    range:
      min: 0
      max: 9223372036854775807
      type: "Long"
    description: "Interval of the object measurements"
  - name: "DebouncePeriod"
    range:
      min: 0
      max: 9223372036854775807
      type: "Long"
    description: "Debounce Period"
  - name: "Emissivity"
    range:
      min: 6553
      max: 65535
      type: "Integer"
    description: "Emissivity"
  - name: "AmbientTemperatureCallbackThreshold"
    complexTypeRef: "TemperatureCallbackThreshold"
    description: "Theshold object for the ambient temperature"
  - name: "ObjectTemperatureCallbackThreshold"
    complexTypeRef: "TemperatureCallbackThreshold"
    description: "Theshold object for the object temperature"
eventDescription:
  events:
  - name: "ObjectTemp"
    range:
      min: -70.0
      max: 380.0
      type: "Double"
    description: "Measured with IR sensor in Celsius"
  - name: "AmbientTemp"
    range:
      min: -40.0
      max: 125.0
      type: "Double"
    description: "Ambient temperature in Celsius"
commandDescription:
  commands:
  - name: "setAmbientTemperatureCallbackPeriod"
    linkedState: "AmbientTemperatureCallbackPeriod"
    parameter:
      CallbackPeriod:
        min: 0
        max: 9223372036854775807
        type: "Long"
  - name: "setObjectTemperatureCallbackPeriod"
    linkedState: "ObjectTemperatureCallbackPeriod"
    parameter:
      CallbackPeriod:
        min: 0
        max: 9223372036854775807
        type: "Long"
  - name: "setEmissivity"
    linkedState: "Emissivity"
    parameter:
      Emissity:
        min: 6553
        max: 65535
        type: "Integer"
  - name: "setAmbientTemperatureCallbackThreshold"
    parameter:
      Threshold: "TemperatureCallbackThreshold"
complexTypes:
- name: "TemperatureCallbackThreshold"
  properties:
  - name: "option"
    description: "'x' :Listener is turned off 
      \ 'o': Listener is triggered when the ambient temperature is outside the min and max values
      \ 'i': Listener is triggered when the ambient temperature is inside the min and max values
      \ '<': Listener is triggered when the ambient temperature is smaller than the min value 
      \ (max is ignored) 
      \ '>' Listener is triggered when the ambient temperature is greater than the min value
      \ (max is ignored)"
    type: "String"
  - name: "min"
    description: "Minimal Value"
    type: "Short"
  - name: "max"
    description: "Maximal Value"
    type: "Short"

\end{minted}

\begin{listing}[H]
\caption{YAML Device Description Temperatur IR Bricklet}
\end{listing}


\textbf{Dualbutton Bricklet}

\begin{minted}[frame=single,
               framesep=3mm,
               linenos=false,
               xleftmargin=21pt,
               tabsize=4,
               fontsize=\footnotesize]{yaml}

---
id: "Dual Button Bricklet"
version: "0.0.1"
stateDescription:
  states:
  - name: "ButtonLeftPressed"
    options:
      values:
      - true
      - false
      type: "Boolean"
    description: "Left Button presses (true) or releases (false)"
  - name: "ButtonRightPressed"
    options:
      values:
      - true
      - false
      type: "Boolean"
    description: "Right Button presses (true) or releases (false)"
  - name: "LedLeft"
    options:
      values:
      - true
      - false
      type: "Boolean"
    description: "Left LED on (true) or off (false)"
  - name: "LedRight"
    options:
      values:
      - true
      - false
      type: "Boolean"
    description: "Right LED on (true) or off (false)"
eventDescription:
  events:
  - name: "ButtonL"
    options:
      values:
      - "Pressed"
      - "Released"
      type: "String"
    description: "State of the left button"
  - name: "ButtonR"
    options:
      values:
      - "Pressed"
      - "Released"
      type: "String"
    description: "State of the right button"
commandDescription:
  commands:
  - name: "setLedL"
    description: "2: LED on, 3: LED off"
    parameter:
      state:
        values:
        - 2
        - 3
        type: "Short"
  - name: "setLedR"
    description: "2: LED on, 3: LED off"
    parameter:
      state:
        values:
        - 2
        - 3
        type: "Short"
complexTypes: []

\end{minted}
\begin{listing}[H]
\caption{YAML Device Description Dual Button Bricklet}
\end{listing}