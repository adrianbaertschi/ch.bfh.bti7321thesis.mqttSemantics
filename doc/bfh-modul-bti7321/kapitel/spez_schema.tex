\chapter{Spezifikation Device Description}
\label{chap:spez}

Dieses Kapitel beschreibt schematisch, wie eine Device Description aufgebaut ist.

\section{Format}
Die Description wird als \gls{utf8} String im Payload der MQTT Message versendet.
Als Datenformat kann JSON oder YAML verwendet werden. Die Beschreibungen sind Case sensitiv.


\section{Primitive Datentypen}

Die Device Description ist als verschachteltes Objekt aufgebaut. Die Werte der Attribute können entweder ein weiteres Objekt oder einen der folgenden primitiven Datentypen enthalten.

Die Definition der primitiven Datentypen orientiert sich an den Datentypen von Java. \cite{jls:4.2}

\begin{table}[H]
\begin{tabular}{ |l|l|r|r| }

 \hline \rowcolor{lightgray}
 {\bf Datentyp } & {\bf Beschreibung } & {\bf Minimun } & {\bf Maximum } \\  \hline


 Integer  &   32 Bit ganzzahlig, signed     &  $-2^{31}$ & $2^{31}$-1  \\ \hline

 Long     &   64 Bit ganzzahlig, signed     &  $-2^{63}$ & $2^{63}$-1  \\ \hline
 
 Float    &   32-bit \gls{ieee_754} floating point, single-precision & $1.4×10^{-45}$  & $3.4028235×10^{38}$  \\ \hline

 Double   &   64-bit \gls{ieee_754} floating point, double-precision & $4.9×10^{-324}$  & $7976931348623157×10^{308}$  \\ \hline
 
 String   &   \gls{utf8} String &   &   \\ \hline
 
\end{tabular}
\caption{Primitive Datentypen}
\end{table}


\section{Schema}

Nachfolgend werden die Felder der Device Description aufgeführt. Bei Plichtfeldern ist der Feldname kursiv formatiert.

\subsection{DeviceDescription Objekt}

\begin{table}[H]
\begin{tabularx}{\textwidth}{|l|l|X|}

 \hline
 {\bf Feld } & {\bf Datentyp } & {\bf Beschreibung } \\  \hline

 \textit{id}  &   String   & Identifikation des Devices   \\ \hline

 \textit{version} & String & API Version des Devices \\ \hline

 description & String & Allgemeine Beschreibung des Devices \\ \hline 
 
 \textit{stateDescription}  &   StateDescription    &     \\ \hline
 
 \textit{eventDescription}  &   EventDescription    &     \\ \hline
  
 \textit{commandDescription}  &   CommandDescription    &     \\ \hline
 
 \textit{complexTypes}  &   Liste ComplexTypes    &     \\ \hline
 
\end{tabularx}
\caption{DeviceDescription Objekt Schema}
\end{table}

\subsection{StateDescription Objekt}
\begin{table}[H]
\begin{tabularx}{\textwidth}{|l|l|X|}

 \hline
 {\bf Feld } & {\bf Datentyp } & {\bf Beschreibung } \\  \hline

 \textit{states}  &   Liste State   & Auflistung von State Objekten   \\ \hline

\end{tabularx}
\caption{StateDescription Objekt Schema}
\end{table}

\subsubsection{State Objekt}
\begin{table}[H]
\begin{tabularx}{\textwidth}{|l|l|X|}

 \hline
 {\bf Feld } & {\bf Datentyp } & {\bf Beschreibung } \\  \hline

 \textit{name}  &   String   & Bezeichung des State Eintrages. Wird gleichzeitig als Subtopic für den eigentlichen Wert genutzt.  \\ \hline
 range  &   Range   &  \textbf Information zum Wert des State.   \\ \hline
 options  &   Enum   & Wird verwendet, falls der State Wert eine Auswahl aus einer fixen Menge ist.   \\ \hline
 complexTypeRef  &   String   & Falls der Wert des State mit einen komplexen Typen abgebildet wird, wird mit diesem Feld der Name des Typs angegeben.   \\ \hline
 description  &   String   &  Allgemeine Beschreibung des State.  \\ \hline

\end{tabularx}
\caption{State Objekt Schema}
\end{table}


\subsection{EventDescription Objekt}
\begin{table}[H]
\begin{tabularx}{\textwidth}{|l|l|X|}

 \hline
 {\bf Feld } & {\bf Datentyp } & {\bf Beschreibung } \\  \hline

 \textit{events}  &   Liste Event   & Auflistung von Event Objekten   \\ \hline

\end{tabularx}
\caption{EventDescription Objekt Schema}
\end{table}


\subsubsection{Event Objekt}
\begin{table}[H]
\begin{tabularx}{\textwidth}{|l|l|X|}

 \hline
 {\bf Feld } & {\bf Datentyp } & {\bf Beschreibung } \\  \hline
 
 \textit{name}  &   String   &  Name des Events. Wird als Subtopic verwendet, auf dem das Event verschickt wird. \\ \hline
 range  &   Range   &  Typinformationen und ev. Einschränkungen, welche den Wert des Events beschreiben   \\ \hline
 options  &   Enum   &  Wird verwendet, falls der Event Wert eine Auswahl aus einer fixen Menge ist.  \\ \hline
 description  &   String   &  Beschreibung des Events  \\ \hline
 complexTypeRef  &   String   &  Falls der Wert des Events mit einen komplexen Typen abgebildet wird, wird mit diesem Feld der Name des Typs angegeben.  \\ \hline

\end{tabularx}
\caption{Event Objekt Schema}
\end{table}



\subsection{CommandDescription Objekt}
\begin{table}[H]
\begin{tabularx}{\textwidth}{|l|l|X|}

 \hline
 {\bf Feld } & {\bf Datentyp } & {\bf Beschreibung } \\  \hline
 \textit{commands}  &   Liste Command   & Auflistung von Command Objekten   \\ \hline

\end{tabularx}
\caption{CommandDescription Objekt Schema}
\end{table}


\subsubsection{Command Objekt}
\begin{table}[H]
\begin{tabularx}{\textwidth}{|l|l|X|}

 \hline
 {\bf Feld } & {\bf Datentyp } & {\bf Beschreibung } \\  \hline
 
 \textit{name}  &   String   &  Name des Commands. Subtopic, an welches der Command gesendet werden muss. \\ \hline
 linkedState  &   String   &  Gibt an, welcher State duch das senden des Commands beeinflusst werden kann. \\ \hline
 description  &   String   &  Beschreibung, was mit dem Command ausgelöst werden kann.  \\ \hline
 \textit{parameter}       &   Map Parameter & Map mit Parameter Objekten. Key ist der Name des Parameters, die der Value ist entweder ein Range, Enum, oder ComplexType Objekt.\\ \hline
\end{tabularx}
\caption{Command Objekt Schema}
\end{table}




\subsection{Range Objekt}

\begin{table}[H]
\begin{tabularx}{\textwidth}{|l|l|X|}

 \hline
 {\bf Feld } & {\bf Datentyp } & {\bf Beschreibung } \\  \hline

 \textit{type}  &  String   & Angabe des primitiven Datentypen        \\ \hline
 \textit{min}   &  gleich wie type   & Minimaler Wert des Bereiches   \\ \hline
 \textit{max}   &  gleich wie type   & Minimaler Wert des Bereiches   \\ \hline

\end{tabularx}
\caption{Range Objekt Schema}
\end{table}

\subsection{Enum Objekt}

\begin{table}[H]
\begin{tabularx}{\textwidth}{|l|l|X|}

 \hline
 {\bf Feld } & {\bf Datentyp } & {\bf Beschreibung } \\  \hline

 values  &   Liste von primitiven Typen   & Liste mit den möglichen Werten   \\ \hline

\end{tabularx}
\caption{Enum Objekt Schema}
\end{table}

