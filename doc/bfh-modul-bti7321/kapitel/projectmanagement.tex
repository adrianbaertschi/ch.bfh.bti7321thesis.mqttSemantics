\chapter{Projektmanagement}
\label{chap:projectmanagement}

\section{Organisation}

\begin{table}[H]
\begin{tabularx}{\textwidth}{|l|l|X|}

 \hline \rowcolor{lightgray}
 {\bf Name } & {\bf Rolle } & {\bf Aufgaben} \\  \hline
  \parbox[t]{5cm}{\textbf{Prof. Dr. Ing. Reto E. König} \\ \textit{Berner Fachhochschule}}  &   Betreuer  &
   für Studierenden, verantwortlich für den Ablauf der Thesis. Beurteilung aufgrund von Aufgabenstellung und abgegebenen Artefakten.   \\
 \hline
  \parbox[t]{5cm}{\textbf{Dr. Federico Flueckiger} \\ \textit{Eidg.  Finanzdepartement}} &   Experte     &
  Beurteilung aufgrund der Aufgabenstellung und abgelieferten Artefakten sowie mindestens ein bis zwei Sitzungen mit dem Studierenden. \\ 
\hline
 \textbf{Adrian Bärtschi}                &   Studierender   &     
 Selbständiges Projektmanagement während der Thesis. Setzt die Aufgaben gemäss Aufgabenstellung und Vorgaben des Betreuers um. Organisiert Kommunikation mit dem Betreuer und Experten.   \\
 \hline


\end{tabularx}
\caption{Involvierte Personen und deren Aufgaben}
\end{table}


\section{Ziele}
Die Arbeit umfasst gemäss Aufgabenstellung folgende Ziele:
\begin{itemize}
    \item Definition einer Beschreibungssprache für vernetzte Geräte.
    \item Die Beschreibung muss von Menschen und Computern interpretiert werden können.
    \item Die Beschreibung muss die Eingabe- und Ausgabeparameter der Geräte beinhalten.
    \item Die Beschreibung soll möglichst die aktuelle MQTT Spezifikation (3.1.1) nicht verletzen.
\end{itemize}

\section{Meilensteine}

Die Aufgabenstellung wurde in Meilensteine aufgeteilt, um eine grobe Planung zu erhalten.

\begin{table}[H]
\begin{tabularx}{\textwidth}{|l|X|}

 \hline \rowcolor{lightgray}
 {\bf Datum } & {\bf Meilenstein } \\  \hline
 
 25.09.2015  &   Initialisierung, Vorgehen geklärt      \\ \hline
 01.10.2015  &   Ziele definiert      \\ \hline
 15.10.2015  &   Einfacher Prototyp erstellt, Konzept der Lösung skizziert     \\ \hline
 30.11.2015  &   Spezifikation Device Description festgelegt     \\ \hline
 15.12.2015  &   Implementation Demo Applikation abgeschlossen     \\ \hline
 18.01.2016  &   Dokumentation inhaltlich abgeschlossen      \\ \hline
 21.01.2016  &   Dokumentation fertiggestellt, Präsentation vorbereitet     \\ \hline
 
\end{tabularx}
\caption{Meilensteine der Thesis}
\end{table}


\section{Ressourcen}

Die gesamte Thesis umfasst gemäss Modulplan der BFH einen Arbeitsaufwand von 360 Stunden. Dies beinhaltet die Konzeption und Umsetzung der Lösung, Absprachen mit Betreuer und Experte, das Erstellen der Dokumentation und die Vorbereitung der Präsentationen für den Finaltag und die Verteidigung.

Es sind keine Kosten für Softwarelizenzen oder andere Ressourcen angefallen.


\section{Aufwände}
Nachfolgend sind die einzelnen einzelnen Tasks der Thesis aufgeführt. Ausserdem werden bei grösseren Differenzen zwischen geplanten und geleisteten Stunden die Gründe dafür angegeben.

\begin{table}[H]
\begin{tabularx}{\textwidth}{|l|r|r|X|}

 \hline \rowcolor{lightgray}
 {\bf Task } & { \bf Geplant [h] } & {\bf Ist [h] }  & {\bf Bemerkungen } \\  \hline
 \textbf{Projektmanagement}         &      &       &   \\ \hline
 Kickoff, Initialisierung           &   8  &   8   &   \\ \hline
 Ziele definieren                   &   6  &   8   &   \\ \hline
 Planung Tasks                      &   8  &  10   &   \\ \hline
 Meetings Betreuer                  &  10  &  12   &   \\ \hline
 Meetings Experte                   &   4  &   2   &  Nur ein Meeting nötig \\ \hline
     &      &       &   \\ \hline
 \textbf{Umsetzung}                 &      &       &   \\ \hline
 MQTT Device Description Library    &  32  &  28   &   \\ \hline
 Tinkerforge Prototyp Java          &  50  &  52   &   \\ \hline
 Webapplikation Device Browser      &  40  &  42   &   \\ \hline
     &      &       &   \\ \hline
 \textbf{Dokumentation}             &      &       &   \\ \hline
 Setup                              &   4  &   8   &  Probleme mit Online Latex Umgebung \\ \hline
 Projektmanagement                  &  16  &  20   &   \\ \hline
 Einleitung, bestehende Umsetzungen &  32  &  30   &   \\ \hline
 Konzept und Anforderungen          &  16  &  18   &   \\ \hline
 Spezifikation Device Description   &  16  &  12   &   \\ \hline
 Umsetzung                          &  32  &  36   &   \\ \hline
 Fazit                              &   8  &   8   &   \\ \hline
 Finalisierung                      &  24  &  22   &   \\ \hline
     &      &       &   \\ \hline
 \textbf{Präsentation}              &      &       &   \\ \hline
 Poster                             &   8  &   8   &   \\ \hline
 Book Seite                         &   8  &  11   &  Mehrere Reviews, Korrekturen nötig\\ \hline
 Präsentation Finaltag              &  24  &  18   &   \\ \hline
 Präsentation Verteidigung          &  14  &  10   &   \\ \hline
     &      &       &   \\ \hline
     &      &       &   \\ \hline
 \textbf{Total}                     & \textbf{360}  &  \textbf{363}     &   \\ \hline

\end{tabularx}
\caption{Planung der Tasks und Auswertung der geleisteten Stunden}
\end{table}


\section{Termine / Abgabefristen}
\begin{table}[H]
\begin{tabularx}{\textwidth}{|l|X|}

 \hline \rowcolor{lightgray}
 {\bf Datum } & {\bf Beschreibung } \\  \hline
 04.01.2016   & Abgabe elektronische Form des Posters  \\ \hline
 11.01.2016   & Erfassung und Freigabe der Book Seite  \\ \hline
 21.01.2016   & Abgabe Dokumentation an Betreuer, Experte und BFH  \\ \hline
 22.01.2016   & Finaltag Bern, Präsentation und Ausstellung \newline
                Abgabe Dokumentation gedruckt und Source Code an Sekretariat BFH  \\ \hline
 01.02.2016   & Verteidigung \\ \hline

\end{tabularx}
\caption{Termine und Fristen}
\end{table}

\section{Ablage}
Sämtliche Ergebnisse der Thesis sind auf dem Github Repository unter \\ \url{https://github.com/barta3/ch.bfh.bti7321thesis.mqttSemantics} verfügbar.

\section{Reflektion}

Dieser Abschnitt blickt auf das Vorgehen während der Thesis zurück und beschreibt kurz die wichtigsten Eindrücke.

Da die Aufgabenstellung relativ offen formuliert ist, war es zu Beginn der Arbeit schwierig sich vorzustellen wie die Lösung aussehen soll und was sie alles beinhaltet. In dieser Phase haben vor allem die regelmässigen Besprechungen mit Reto Koenig geholfen, um Ideen zu diskutieren und jeweils die nächsten Schritte zu planen.

Aufgrund der geringen Verbreitung und Bekanntheit des MQTT Protokolls sind keine etablierten Lösungen vorhanden, von deren Erfahrungen man profitieren konnte. Dies bot jedoch auch eine grosse Freiheit bei der Gestaltung und Umsetzung der eigenen Lösung.

Beim Entwurf der Device Description und der Umsetzung hat sich gezeigt, dass sich ein iteratives Vorgehen   gut eignet. Mit der frühen Entwicklung von Prototypen und ständiger Verbesserung und Anpassung konnte die Aufgabenstellung gut in kleinere Pakete aufgeteilt werden.

Da die eingesetzten Devices von Tinkerforge viele verschiedene Interaktionsmöglichkeiten bieten, konnten nicht alle Funktionen mit der Device Description abgebildet werden. Bei Devices, welche einen spezifischeren Einsatzzweck haben, sollten dann die Device Descriptions auch entsprechend kompakter werden.


Für die Erarbeitung der Dokumentation wurde LaTeX verwendet, basierend auf der Vorlage der BFH. Dies hat sich nach anfänglichen Schwierigkeiten bei der Installation als gute Entscheidung erwiesen. Dank der vordefinierten Formatierung konnte ich mich auf die das Erstellen der Inhalte konzentrieren. Ausserden eignete sich diese Dokumentationsform gut für das eingesetzte Versionskontrollsystem (Git).

Je technischer die Dokumentation wurde, desto umständlicher wurde es, die Inhalte verständlich in Deutscher Sprache zu formulieren. Rückblickend wäre es ev. besser gewesen, das Dokument in Englisch zu verfassen, vor allem um die technischen Begriffe einheitlicher verwenden zu können.
