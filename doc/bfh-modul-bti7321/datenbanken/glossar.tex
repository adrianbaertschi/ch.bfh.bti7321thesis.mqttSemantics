
\newacronym{iot}{IoT}{Internet of Things}

\newacronym{uid}{UID}{Unique Identifier}

\newacronym{a_mqtt}{MQTT}{Masdf}

\newacronym{qos}{QoS}{Quality of Service}

\newglossaryentry{mqtt}
{
    name=MQTT,
    description={Netzwerkprotokoll für IoT Anwenungen}
}


\newglossaryentry{utf8}
{
    name=UTF-8,
    description={Character Encoding für Unicode \url{https://tools.ietf.org/html/rfc3629}}, 
}

\newglossaryentry{ieee_754}
{
    name=IEEE 754,
    description={IEEE Standard for Floating-Point Arithmetic   \url{http://ieeexplore.ieee.org/servlet/opac?punumber=4610933}}, 
}


\newglossaryentry{base64}
{
    name=Base64,
    description={Verfahren zur Codierung von binäredaten in eine Zeichenfolge
     \url{https://tools.ietf.org/html/rfc3548}}
}

\newglossaryentry{serialisierung}
{
    name=Serialisierung,
    description={Strukturierte Daten (Objekte) in sequenzielle Form bringen zur Persistierung oder Netzwerkübertragung }, 
}

\newglossaryentry{json}
{
    name=JSON,
    description={JavaScript Object Notation; Kompaktes, textbasiertes Datenformat
    \url{https://tools.ietf.org/html/rfc7159}}, 
}

\newglossaryentry{yaml}
{
    name=YAML,
    description={YAML Ain't Markup Language; Gut lesbares, textbasiertes Datenformat
    \url{http://www.yaml.org/spec/1.2/spec.html}},
}