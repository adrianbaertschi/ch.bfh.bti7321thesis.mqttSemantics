\chapter{Spezifikation Device Description}
\label{chap:spez}



\section{Device Description}

d
% http://swagger.io/specification/

\subsection{Codierung}
UTF-8


\section{Format}
Json, Yaml

TODO: case sensitiv?


\section{Primitive Datentypen}

\begin{table}[h!]
\begin{tabular}{ |l|l|r|r| }

 \hline
 {\bf Datentyp } & {\bf Beschreibung } & {\bf Minimun } & {\bf Maximum } \\  \hline


 Integer  &   32 Bit ganzzahlig, signed     &  $-2^{31}$ & $2^{31}$-1  \\ \hline

 Long     &   64 Bit ganzzahlig, signed     &  $-2^{63}$ & $2^{63}$-1  \\ \hline
 
 Float    &   32-bit \gls{ieee_754} floating point, single-precision & $1.4×10^{-45}$  & $3.4028235×10^{38}$  \\ \hline

 Double   &   64-bit \gls{ieee_754} floating point, double-precision & $4.9×10^{-324}$  & $7976931348623157×10^{308}$  \\ \hline
 
 String   &   \gls{utf8} String &   &   \\ \hline
 
 Boolean   &  Mögliche Werte: true / false &   &   \\ \hline
 
\end{tabular}
\caption{Primitive Datentypen}
\end{table}



\section{DeviceDescription Objekt}

Hintweis optionale Felder

\begin{table}[h!]
\begin{tabularx}{\textwidth}{|l|l|X|}

 \hline
 {\bf Feld } & {\bf Datentyp } & {\bf Beschreibung } \\  \hline

 id  &   String   & Identifikation des Devices   \\ \hline

 version & String & Version API des Devices \\ \hline

 description & String & Allgemeine Beschreibung des Devices \\ \hline 
 
 stateDescription  &   StateDescription    &     \\ \hline
 
 eventDescription  &   EventDescription    &     \\ \hline
  
 commandDescription  &   CommandDescription    &     \\ \hline
 
 complexTypes  &   ComplexTypes    &     \\ \hline
 
\end{tabularx}
\caption{DeviceDescription Objekt Schema}
\end{table}

\subsection{StateDescription}
\begin{table}[h!]
\begin{tabularx}{\textwidth}{|l|l|X|}

 \hline
 {\bf Feld } & {\bf Datentyp } & {\bf Beschreibung } \\  \hline

 states  &   Liste State   & Auflistung von State Objekten   \\ \hline

\end{tabularx}
\caption{StateDescription Objekt Schema}
\end{table}

\subsubsection{State}
\begin{table}[h!]
\begin{tabularx}{\textwidth}{|l|l|X|}

 \hline
 {\bf Feld } & {\bf Datentyp } & {\bf Beschreibung } \\  \hline

 name  &   String   & \textbf{Pflichtfeld.} Bezeichung des State-Eintrages. Wird gleichzeitig als Subtopic für den eigentlichen Wert genutzt.  \\ \hline
 range  TODO:naming &   Range   &  \textbf{Plichtfeld.} Information zum Wert des State.   \\ \hline
 options  &   Enum   & Wird verwendet, falls der State Wert eine Auswahl aus einer fixen Menge ist.   \\ \hline
 complexTypeRef  &   String   & Falls der Wert des State mit einen komplexen Typen abgebildet wird, wird mit diesem Feld der Name des Typs angegeben. siehe TODO   \\ \hline
 description  &   String   &  Allgemeine Beschreibung des State.  \\ \hline

\end{tabularx}
\caption{State Objekt Schema}
\end{table}


\subsection{EventDescription}
\begin{table}[h!]
\begin{tabularx}{\textwidth}{|l|l|X|}

 \hline
 {\bf Feld } & {\bf Datentyp } & {\bf Beschreibung } \\  \hline

 events  &   Liste Event   & Auflistung von Event Objekten   \\ \hline

\end{tabularx}
\caption{EventDescription Objekt Schema}
\end{table}


\subsubsection{Event}
\begin{table}[h!]
\begin{tabularx}{\textwidth}{|l|l|X|}

 \hline
 {\bf Feld } & {\bf Datentyp } & {\bf Beschreibung } \\  \hline
 
 name  &   String   &  \textbf{Pflichtfeld.} Name des Events. Wird als Subtopic verwendet, auf dem das Event verschickt wird. \\ \hline
 range  &   Range   &  \textbf{Pflichtfeld.} Typinformationen und ev. Einschränkungen, welche den Wert des Events beschreiben   \\ \hline
 options  &   Enum   &  Wird verwendet, falls der Event Wert eine Auswahl aus einer fixen Menge ist.  \\ \hline
 description  &   String   &  Beschreibung des Events  \\ \hline
 complexTypeRef  &   String   &  Falls der Wert des Events mit einen komplexen Typen abgebildet wird, wird mit diesem Feld der Name des Typs angegeben. siehe TODO  \\ \hline

\end{tabularx}
\caption{Event Objekt Schema}
\end{table}



\subsection{CommandDescription}
\begin{table}[h!]
\begin{tabularx}{\textwidth}{|l|l|X|}

 \hline
 {\bf Feld } & {\bf Datentyp } & {\bf Beschreibung } \\  \hline
 commands  &   Liste Command   & Auflistung von Command Objekten   \\ \hline

\end{tabularx}
\caption{CommandDescription Objekt Schema}
\end{table}


\subsubsection{Command}
\begin{table}[h!]
\begin{tabularx}{\textwidth}{|l|l|X|}

 \hline
 {\bf Feld } & {\bf Datentyp } & {\bf Beschreibung } \\  \hline
 
 name  &   String   &  \textbf{Pflichtfeld.} Name des Commands. Subtopic, an welches der Command gesendet werden muss. \\ \hline
 linkedState  &   String   &  Gibt an, welche States duch das senden des Commands beiinfluss werden können. \\ \hline
 options  &   Enum   &   Wird verwendet, falls der Command eine fix definierte Menge an Werten beinhalten kann. \\ \hline
 Fehlt: description  &   String   &  Beschreibung, was mit dem Command ausgelöst werden kann.  \\ \hline
 complexTypeRef  &   String   &  Falls der Wert des Events mit einen komplexen Typen abgebildet wird, wird mit diesem Feld der Name des Typs angegeben. siehe TODO  \\ \hline

\end{tabularx}
\caption{Command Objekt Schema}
\end{table}







%End (Allgemeine objekte)
\subsection{Range}
asdf
\begin{table}[h!]
\begin{tabularx}{\textwidth}{|l|l|X|}

 \hline
 {\bf Feld } & {\bf Datentyp } & {\bf Beschreibung } \\  \hline

 topic  &   String   &    \\ \hline


\end{tabularx}
\caption{State Description Schema}
\end{table}

\subsection{Enum}
asdf
\begin{table}[h!]
\begin{tabularx}{\textwidth}{|l|l|X|}

 \hline
 {\bf Feld } & {\bf Datentyp } & {\bf Beschreibung } \\  \hline

 topic  &   String   &    \\ \hline


\end{tabularx}
\caption{State Description Schema}
\end{table}

Tabelle mit Felder und Beschreibung