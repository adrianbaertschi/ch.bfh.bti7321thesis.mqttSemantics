\chapter{Fazit}
\label{chap:schlussfolgerungen}

Es hat sich gezeigt, dass allgemein bei MQTT Anwendungen zwei Hauptschwierigkeiten auftreten. Zum einen gibt es keinen Mechanismus für die Angabe resp. das Finden der Topics auf dem Broker. Der zweite Punkt ist das fehlen einer Struktur der Message Payloads. 

Diese Grundprobleme treten auch beim Einsatz von MQTT für vernetzte Geräte auf. Als Ergebis der Arbeit ist ein Konzept entstanden, welches eine allgemeine Lösung für die Beschreibung von Geräten per MQTT darstellt. Die Umsetzung des Prototypen hat gezeigt, dass ein generischer Ansatz möglich ist.

Als nächsten Schritt wäre es interessant zu sehen, wie gut sich das Konzept für einen Anwendungsfall aus der Praxis eignet und welche Anpassungen ev. noch gemacht werden müssten.

Es ist schwierig abzuschätzen, ob sich in dieser Domäne überhaupt je ein Standard bilden und etablieren wird.
Hinweis LWM2C, andere Protokolle, viel neues.