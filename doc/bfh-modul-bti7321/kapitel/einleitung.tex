\chapter{Einleitung}
\label{chap:einleitung}

TODO: überarbeiten

Bei Systemen im \acrfull{iot} Umfeld sind sehr viele und auch unterschiedliche Geräte in einen Netzwerk miteindander verbunden. 
Für diese IoT - Machine-To-Machine Kommunikation werden andere Netzwerkprotokolle eingesetzt als im 'klassischen' Internet. Dies ist nötig, weil die Geräte stark eingeschränkte Resourcen haben und die Netzwerke geringe Bandbreiten aufweisen.

MQTT ist ein Protokoll, welches die Anforderungen für IoT Systeme erfüllen soll. Beim Entwurf des Protokolls wurde auf Einfachheit und Leichtgewichtigkeit grossen Wert gelegt. Mit MQTT ist es möglich, fast beliebige Daten in beliebiger Codierung zu versenden. Dies bietet den Entwicklern der Systeme grosse Freiheiten. \\
Es ist aber ersichtlich, dass die fehlende Struktur und Beschreibung der Daten gewisse Schwierigkeiten mit sich bringen kann. Eine Anwendung, welche Daten per MQTT erhält, muss wissen wie diese vom Absender codiert wurden und was sie bedeuten.

Beispielsweise wird eine Messung eines Temperatursensors via MQTT versendet werden. Der Sensor liefert den Wert 22° Celsius.
Der ganzzahlige Wert 22 wird als binärer Wert \texttt{10110} versendet.
Der Empfänger erhält nun die MQTT Nachricht \texttt{10110}. Er weis aber nicht, was mit diesem Wert anzufangen ist. Wird dieser Wert nicht nach dem gleichen Schema interpretiert, wie der codiert wurde, werden Daten erzeugt, welche nicht der Ursprungsinforamtion entsprechen.
Der Empfänger müsste wissen, dass es sich um einen Integer Wert (hier 32 bit signed) handelt, damit die Daten in das richtige Format gebracht werden können.
Ausserdem muss der Empfänger jetzt noch wissen, in welcher Einheit (Celsius, Fahrenheit, etc.) die Temperatur übermittelt wird und es wäre praktisch zu wissen, wo zum Beispiel die Messung stattfand.

Die Erzeuger der Daten stellen diese Information den Entwicklern von Anwendungen zurzeit typischweise in einnen Dokument zur Verfügung. Diese sogenannte out-out-band Dokumentation ist aber aufwändig in der Nachführung und und Änderungen. Auch ist es schwierig die Struktur der Daten einheitlich und klar zu erklären.

%Ziel dieser Thesis ist es, dass Geräte welche ihre Daten per MQTT versenden, die Möglichkeit erhalten, sich selbst inkusive ihrer Daten und Möglichkeiten zur Interkation zu beschreiben. Dies soll so getan werden, dass die Beschreibung für Mensch und Maschine les- und verstehbar ist und der MQTT Standard weiterhin eigehalten wird.