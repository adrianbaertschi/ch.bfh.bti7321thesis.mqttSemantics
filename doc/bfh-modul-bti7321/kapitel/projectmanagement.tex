\chapter{Project Managament}
\label{chap:projectmanagement}

\section{Organisation}
TODO: formatieren

\begin{table}[h!]
\begin{tabularx}{\textwidth}{|l|l|X|}

 \hline
 {\bf Name } & {\bf Rolle } & {\bf Aufgaben} \\ 
 \hline
 Adrian Bärtschi                &   Studierender   &     
 Selbständiges Projektmanagement während der Thesis. Setzt die Aufgaben gemäss Aufgabenstellung und Vorgaben Betreuer um. Organisiert Kommunikation mit Betreuer und Experte.   \\
 \hline
  \parbox[t]{5cm}{Prof. Dr. Ing. Reto E. König \\ Berner Fachhochschule}  &   Betreuer  &
  Hauptansprechsperson für Studierenden, verantwortlich für den Ablauf der Thesis. Beurteilung aufgrund von Aufgabenstellung und abgegebenen Artefakten.   \\
 \hline
  \parbox[t]{5cm}{Dr. Federico Flueckiger \\ Eidg.  Finanzdepartement} &   Experte     &
  Beurteilung aufgrund der Aufgabenstellung und abgelieferten Artefakten sowie mindestends ein bis zwei Sitzungen mit dem Studierenden. \\ 
\hline

\end{tabularx}
\caption{Involvierte Personen und deren Aufgaben}
\end{table}



\section{Meilensteine}

\begin{table}[h!]
\begin{tabularx}{\textwidth}{|l|X|l|}

 \hline
 {\bf Datum } & {\bf Meilenstein } & {\bf Bemerkungen } \\ 
 \hline
 25.09.2015  &   Initialisierung, Vorgehen geklärt   &     -   \\ \hline
 01.10.2015  &   Ziele definiert   &     -   \\ \hline
 15.10.2015  &   Prototyp erstellt, Konzept der Lösung skizziert   &     -   \\ \hline
 30.11.2015  &   Implementation System abgeschlossen   &     -   \\ \hline
 15.12.2015  &   Implementation Demo Applikation abgeschlossen   &     -   \\ \hline
 18.01.2016  &   Dokumentation inhaltlich abgeschlossen    &     -   \\ \hline
 21.01.2016  &   Dokumentation fertiggestellt    &     -   \\ \hline
 
\end{tabularx}
\caption{Meilensteine der Thesis}
\end{table}


\section{Resourcen}
Kosten, etc.


\section{Ziele}
Die zu entwickelnde Lösung soll folgendes beinhalten:
\begin{itemize}
    \item asdf
\end{itemize}


\section{Aufwände}

\begin{table}[h!]
\begin{tabularx}{\textwidth}{|l|l|l|X|}

 \hline
 {\bf Task } & {\bf Soll } & {\bf Ist }  & {\bf Bemerkungen } \\ 
 \hline
 asdf    &   2   &  2     & df  \\
 \hline

\end{tabularx}
\caption{Planung der Tasks und Auswetung der Aufwände}
\end{table}


\section{Deliverables / Termine}
\begin{table}[h!]
\begin{tabularx}{\textwidth}{|l|X|}

 \hline
 {\bf Datum } & {\bf Thema } \\ 
 \hline
              & Abgabe Texte/Grafiken/etc. für das Book  \\ \hline
              & Abgabe Elektronische Version des Posters  \\ \hline
 21.01.2016   & Abgabe Projektdokumentation an Betreuer, Experte, Sekretariat  \\ \hline
 22.01.2016   & Final Day Bern, Präsentation und Ausstellung  \\ \hline
              & Verteidigung \\ \hline

\end{tabularx}
\caption{Termine und Fristen}
\end{table}