\chapter{Projektmanagament}
\label{chap:projectmanagement}

\section{Organisation}

\begin{table}[H]
\begin{tabularx}{\textwidth}{|l|l|X|}

 \hline \rowcolor{lightgray}
 {\bf Name } & {\bf Rolle } & {\bf Aufgaben} \\  \hline
  \parbox[t]{5cm}{\textbf{Prof. Dr. Ing. Reto E. König} \\ \textit{Berner Fachhochschule}}  &   Betreuer  &
  Hauptansprechsperson für Studierenden, verantwortlich für den Ablauf der Thesis. Beurteilung aufgrund von Aufgabenstellung und abgegebenen Artefakten.   \\
 \hline
  \parbox[t]{5cm}{\textbf{Dr. Federico Flueckiger} \\ \textit{Eidg.  Finanzdepartement}} &   Experte     &
  Beurteilung aufgrund der Aufgabenstellung und abgelieferten Artefakten sowie mindestends ein bis zwei Sitzungen mit dem Studierenden. \\ 
\hline
 \textbf{Adrian Bärtschi}                &   Studierender   &     
 Selbständiges Projektmanagement während der Thesis. Setzt die Aufgaben gemäss Aufgabenstellung und Vorgaben Betreuer um. Organisiert Kommunikation mit Betreuer und Experte.   \\
 \hline


\end{tabularx}
\caption{Involvierte Personen und deren Aufgaben}
\end{table}


\section{Ziele}
Die Arbeit umfasst gemäss Aufgabenstellung folgende Ziele:
\begin{itemize}
    \item Definition einer Beschreibungssprache für vernetzte Geräte
    \item Die Beschreibung muss von Menschen und Computern interpretiert werden können
    \item Die Beschreibung muss die Eingabe- und Ausgabeparameter der Geräte beeinhalten
    \item Die Beschreibung muss soll möglichst die aktuelle MQTT Spezifikation (3.1.1) nicht verletzen
\end{itemize}

\section{Meilensteine}

\begin{table}[H]
\begin{tabularx}{\textwidth}{|l|X|}

 \hline \rowcolor{lightgray}
 {\bf Datum } & {\bf Meilenstein } \\  \hline
 
 25.09.2015  &   Initialisierung, Vorgehen geklärt      \\ \hline
 01.10.2015  &   Ziele definiert      \\ \hline
 15.10.2015  &   Einfacher Prototyp erstellt, Konzept der Lösung skizziert     \\ \hline
 30.11.2015  &   Spezifikation Device Description festgelegt     \\ \hline
 15.12.2015  &   Implementation Demo Applikation abgeschlossen     \\ \hline
 18.01.2016  &   Dokumentation inhaltlich abgeschlossen      \\ \hline
 21.01.2016  &   Dokumentation fertiggestellt, Präsentation vorbereitet     \\ \hline
 
\end{tabularx}
\caption{Meilensteine der Thesis}
\end{table}


\section{Ressourcen}

Die gesamte Thesis umfasst gemäss Modulplan der BFH 360 Stunden. Dies beinhaltet die Konzeption und Umsetzung der Lösung, das Erstellen der Dokumentation und die Vorbereitung der Präsentationen für den Finaltag und die Verteidigung.

Es sind keine Kosten für Softwarelizenzen oder andere Ressourcen angefallen.


\section{Aufwände}
Nachfolgend ist der Gesamtaufand von 360 Stunden der Arbeit auf einzelen Tasks aufgeteilt. Ausserdem werden bei Differenzen zwischen geplanten und geleisteten Stunden die Gründe dafür angegeben.

\begin{table}[H]
\begin{tabularx}{\textwidth}{|l|r|r|X|}

 \hline \rowcolor{lightgray}
 {\bf Task } & { \bf Geplant [h] } & {\bf Ist [h] }  & {\bf Bemerkungen } \\  \hline
 \textbf{Projektmanagement}         &      &       &   \\ \hline
 Kickoff, Initialisierung           &   8  &   8   &   \\ \hline
 Ziele definieren                   &   6  &   8   &   \\ \hline
 Planung Tasks                      &   8  &  10   &   \\ \hline
 Meetings Betreuer                  &  10  &  12   &   \\ \hline
 Meetings Experte                   &   4  &   2   &   \\ \hline
     &      &       &   \\ \hline
 \textbf{Umsetzung}                 &      &       &   \\ \hline
 MQTT Device Description Library    &  32  &       &   \\ \hline
 Tinkerforge Prototyp Java          &  50  &       &   \\ \hline
 Webapplikation Device Browser      &  40  &       &   \\ \hline
     &      &       &   \\ \hline
 \textbf{Dokumentation}             &      &       &   \\ \hline
 Setup                              &   4  &       &   \\ \hline
 Projektmanagement                  &  16  &       &   \\ \hline
 Einleitung, bestehende Umsetzungen &  32  &       &   \\ \hline
 Konzept und Anforderungen          &  16  &       &   \\ \hline
 Spezifikation Device Description   &  16  &       &   \\ \hline
 Umsetzung                          &  32  &       &   \\ \hline
 Fazit                              &   8  &       &   \\ \hline
 Finalisierung                      &  24  &       &   \\ \hline
     &      &       &   \\ \hline
 \textbf{Präsentation}              &      &       &   \\ \hline
 Poster                             &   8  &       &   \\ \hline
 Book Seite                         &   8  &       &   \\ \hline
 Präsentation Finaltag              &  24  &       &   \\ \hline
 Präsentation Verteidigung          &  14  &       &   \\ \hline
     &      &       &   \\ \hline
     &      &       &   \\ \hline
 \textbf{Total}                     & \textbf{360}  &       &   \\ \hline
     
 \hline

\end{tabularx}
\caption{Planung der Tasks und Auswertung der geleisteten Stunden}
\end{table}


\section{Termine / Abgabefristen}
\begin{table}[H]
\begin{tabularx}{\textwidth}{|l|X|}

 \hline \rowcolor{lightgray}
 {\bf Datum } & {\bf Beschreibung } \\  \hline
 04.01.2016   & Abgabe elektronische Form des Posters  \\ \hline
 11.01.2016   & Erfassung und Freigabe der Book Seite  \\ \hline
 21.01.2016   & Abgabe Projektdokumentation an Betreuer, Experte und Sekretariat  \\ \hline
 22.01.2016   & Finaltag Bern, Präsentation und Ausstellung  \\ \hline
 01.02.2016   & Verteidigung \\ \hline

\end{tabularx}
\caption{Termine und Fristen}
\end{table}

\section{Reflektion}
nicht klare ziele, ungewissheit
Freiheit


+ Prototyping
+ Frühe Entwicklung
+ Latex dok


- Deutsche Doku
