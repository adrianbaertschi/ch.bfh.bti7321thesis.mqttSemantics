\chapter{Fazit}
\label{chap:schlussfolgerungen}

Es hat sich gezeigt, dass allgemein bei MQTT Anwendungen zwei Hauptschwierigkeiten auftreten. Zum einen gibt es keinen Mechanismus für die Angabe resp. das Finden (Discovery) der Topics auf dem Broker. Der zweite Punkt ist die fehlende Struktur der Message Payloads. 

Diese Grundprobleme treten auch beim Einsatz von MQTT für vernetzte Geräte in Erscheinung. Als Ergebnis der Arbeit ist ein Konzept entstanden, welches mithilfe einer definierten Topic Struktur und einem Schema für die Device Description einen möglichen Lösungsweg aufzeigt. Die Umsetzung eines Prototyps hat gezeigt, dass das Konzept grundsätzlich funktionstüchtig ist, jedoch noch Potenzial für Verbesserungen aufweist.

Als nächster Schritt wäre es interessant zu sehen, wie gut sich die entwickelten Ideen für einen konkreten Anwendungsfall aus der Praxis eignen und welche Anpassungen dafür ev. noch gemacht werden müssten.


Es gilt zu beachten, dass das Thema Internet of Things noch neu und stark in Bewegung ist. Ob sich ein einheitliches Konzept für die Beschreibung von Devices per MQTT einmal durchsetzen wird, ist momentan schwer abzuschätzen. Obwohl der Wunsch nach einheitlichen Schnittstellen vorhanden ist, gibt  es zurzeit wenig Anzeichen, dass sich ein Standard bilden und etablieren wird.