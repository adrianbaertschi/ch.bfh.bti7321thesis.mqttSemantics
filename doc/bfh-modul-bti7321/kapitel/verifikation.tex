\chapter{Verifikation}
\label{chap:verification}

\section{Funktionale Anforderungen}
Anhand der definierten Use Cases wird getestet, ob die funktionalen Anforderungen erfüllt sind.


\begin{table}[H]
\begin{tabularx}{\textwidth}{|l|X|c|}

 \hline \rowcolor{lightgray}
 {\bf Use Case } & {\bf Bewertung } & {\bf Ergebnis} \\  \hline
 
 UC01: Beschreibung Device & Entwickeltes Schema enthält die geforderten Angaben. & OK \\ \hline

 UC02: Angabe Datentypen   & Die Beschreibungsobjekte können mit Datentypen vesehen werden. & OK \\ \hline

 UC03: Definition Range    & Die Datentypen können mit Ranges eingeschränkt werden. & OK \\ \hline

 UC04: Definition Enum     & Die Beschreibung unterstützt Datentypen mit fixen Auswahllisten & OK \\ \hline

 UC05: Gruppierung Devices & Die Devices können auf verschiedenen Stufen gruppiert werden. & OK \\ \hline

 UC06: Konfiguration Device Description Library & Die Library ist so aufgebaut, dass  Konfigurationsparamter gesetzt werden können. & OK \\ \hline

 UC07: Anzeige Devices & Devices werden in der Webapplikation angezeigt. & OK \\ \hline

 UC08: Anzeige Device Description & Description eines Devices wird in Klartext und in interpretierter Form in der Webapplikation angezeigt. & OK \\ \hline

 UC09: Anzeige Eventdaten eines Devices & Der Benutzer kann sich über die Webapplikation Eventdaten anzeigen lassen. & OK \\ \hline

 UC10: Versenden eines Commands & Der Benutzer kann über die Webapplikation Commands erfassen und an die Devices senden. & OK \\ \hline

\end{tabularx}
\caption{Verifikation funktionale Anforderungen}
\end{table}





\section{Nichtfunktionale Anforderungen}

\begin{table}[H]
\begin{tabularx}{\textwidth}{|l|X|c|}

 \hline \rowcolor{lightgray}
 {\bf Anforderung } & {\bf Bewertung } & {\bf Ergebnis} \\  \hline
 
 Erweiterbarkeit Devices & Mit der Modularisierung der Device Description Library ist die Abgrenzung zu den konkreten Umsetzungen sichergestellt. & OK \\ \hline

 Erweiterbarkeit Datenformate   & Durch den gewählten Aufbau der Topic Hierarchie ist es möglich, verschiedene Formate für die Device Description zu verwenden. & OK \\ \hline

 Einfache Installation    & Die Device Description Library kann als Maven Modul in bestehende Anwendungen integriert werden. Die Wepplikation kann ohne Abhängigkeiten intalliert werden, die Konfiguration ist zentral definiert. & OK \\ \hline

 Kompatibilität     & Die entwickelte Lösung besiert auf dem bestehenden MQTT Protokoll und ist somit kompatibel für die Anbindung an beliebigen Applikationen auf Basis von MQTT. & OK \\ \hline


\end{tabularx}
\caption{Verifikation nichtfunktionale Anforderungen}
\end{table}
