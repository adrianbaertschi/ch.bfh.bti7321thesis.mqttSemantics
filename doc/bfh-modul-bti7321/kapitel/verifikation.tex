\chapter{Verifikation}
\label{chap:verification}

\section{Funktionale Anforderungen}
Anhand der definierten Use Cases wird getestet, ob die funktionalen Anforderungen erfüllt sind.

\begin{table}[H]
\begin{tabularx}{\textwidth}{|l|X|c|}

 \hline \rowcolor{lightgray}
 {\bf Use Case } & {\bf Bewertung } & {\bf Ergebnis} \\  \hline
 
 UC01 - Device beschreiben & Entwickeltes Schema enthält die geforderten Angaben. & OK \\ \hline

 UC02 - Datentypen angeben   & Die Beschreibungsobjekte können mit Datentypen versehen werden. & OK \\ \hline

 UC03 - Range definieren    & Die Datentypen können mit Ranges eingeschränkt werden. & OK \\ \hline

 UC04 - Enum definieren     & Die Beschreibung unterstützt Datentypen mit fixen Auswahllisten & OK \\ \hline

 UC05 - Devices gruppieren & Die Devices können auf verschiedenen Stufen gruppiert werden. & OK \\ \hline

 UC06 - Device Description Library konfigurieren & Die Library ist so aufgebaut, dass  Konfigurationsparameter gesetzt werden können. & OK \\ \hline

 UC07 - Devices anzeigen & Devices werden in der Webapplikation angezeigt. & OK \\ \hline

 UC08 - Device Description anzeigen & Description eines Devices wird in Klartext und in interpretierter Form in der Webapplikation angezeigt. & OK \\ \hline

 UC09 - Eventdaten eines Devices anzeigen & Der Benutzer kann sich über die Webapplikation Eventdaten anzeigen lassen. & OK \\ \hline

 UC10 - Commands an Device senden & Der Benutzer kann über die Webapplikation Commands erfassen und an die Devices senden. & OK \\ \hline

\end{tabularx}
\caption{Verifikation funktionale Anforderungen}
\end{table}





\section{Nicht-funktionale Anforderungen}

\begin{table}[H]
\begin{tabularx}{\textwidth}{|l|X|c|}

 \hline \rowcolor{lightgray}
 {\bf Anforderung } & {\bf Bewertung } & {\bf Ergebnis} \\  \hline
 
 Erweiterbarkeit Devices & Mit der Modularisierung der Device Description Library ist die Abgrenzung zu den konkreten Umsetzungen sichergestellt. & OK \\ \hline

 Erweiterbarkeit Datenformate   & Durch den gewählten Aufbau der Topic Hierarchie ist es möglich, verschiedene Formate für die Device Description zu verwenden. & OK \\ \hline

 Einfache Installation    & Die Device Description Library kann als Maven Modul in bestehende Anwendungen integriert werden. Die Webapplikation kann ohne Abhängigkeiten installiert werden, die Konfiguration ist zentral definiert. & OK \\ \hline

 Kompatibilität     & Die entwickelte Lösung basiert auf dem bestehenden MQTT Protokoll und ist somit kompatibel für die Anbindung an beliebigen Applikationen auf Basis von MQTT. & OK \\ \hline


\end{tabularx}
\caption{Verifikation nicht-funktionale Anforderungen}
\end{table}




\section{Verbesserungsmöglichkeiten}

Während der Umsetzung und dem Test des Systems wurden die folgenden drei Hauptpunkte identifiziert, bei denen das grösste Verbesserungspotenzial vorliegt.

\subsection{Device Description: Darstellung Typen}
Bei der Darstellung der Device Description war die grösste Schwierigkeit die Abbildung von Typinformationen. Die aktuelle Lösung mit drei verschiedenen Möglichkeiten zur Beschreibung von Typen (Range, Enum und ComplexType) sollte vereinfacht werden.

\subsection{Description Library: Integration Deserialisierung}
In der aktuellen Implementation der MQTT Device Description Library werden die Payload Daten der empfangenen Commands als Byte Array an den Aufrufer der Library geliefert.
Da die Library das Schema des empfangenen Commands kennt, würde es Sinn machen, die Deserialisierung der Daten in die Library zu integrieren und somit dem Aufrufer das erstellte Objekte zu übergeben.

\subsection{Device Description an Devicetyp anhängen}
In der Hierarchie der MQTT Topics ist die Device Description momentan einer Device Instanz untergeordnet. Bei vielen Devices in der selben Gruppierung vom gleichen Typ würde dies zu Redundanz und unnötig grossen Datenmengen führen, da die gleiche Device Description für jede Instanz versendet würde. Um dies zu optimieren, müsste die Device Description unterhalb des Devicetyps in die Schema Hierarchie eingegliedert werden. Mit diesem Ansatz wäre pro Devicetyp nur noch das Versenden von einer Device Description nötig.
