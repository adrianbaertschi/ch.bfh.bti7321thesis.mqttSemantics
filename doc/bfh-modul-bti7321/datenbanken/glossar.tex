
\newacronym{api}{API}{Application Program Interface}

\newacronym{iot}{IoT}{Internet of Things}

\newacronym{uid}{UID}{Unique Identifier}

\newacronym{a_mqtt}{MQTT}{Message Queue Telemetry Transport}

\newacronym{qos}{QoS}{Quality of Service}

\newacronym{sdk}{SDK}{Software Development Kit}



\newglossaryentry{mqtt}
{
    name=MQTT,
    description={Netzwerkprotokoll für IoT Anwendungen}
}


\newglossaryentry{utf8}
{
    name=UTF-8,
    description={Character Encoding für Unicode \url{https://tools.ietf.org/html/rfc3629}}, 
}

\newglossaryentry{ieee_754}
{
    name=IEEE 754,
    description={IEEE Standard for Floating-Point Arithmetic   \url{http://ieeexplore.ieee.org/servlet/opac?punumber=4610933}}, 
}


\newglossaryentry{base64}
{
    name=Base64,
    description={Verfahren zur Codierung von Binärdaten in eine Zeichenfolge
     \url{https://tools.ietf.org/html/rfc3548}}
}

\newglossaryentry{serialisierung}
{
    name=Serialisierung,
    description={Strukturierte Daten (Objekte) in sequenzielle Form bringen zur Persistierung oder Netzwerkübertragung }, 
}

\newglossaryentry{json}
{
    name=JSON,
    description={JavaScript Object Notation; Kompaktes, textbasiertes Datenformat
    \url{https://tools.ietf.org/html/rfc7159}}, 
}

\newglossaryentry{yaml}
{
    name=YAML,
    description={YAML Ain't Markup Language; Gut lesbares, textbasiertes Datenformat
    \url{http://www.yaml.org/spec/1.2/spec.html}}
}

\newglossaryentry{xml}
{
    name=XML,
    description={Extensible Markup Language \url{https://www.w3.org/TR/2006/REC-xml11-20060816/}}
}

\newglossaryentry{soap}
{
    name=SOAP,
    description={Simple Object Access Protocol \url{https://www.w3.org/TR/soap12/}}
}

\newglossaryentry{wsdl}
{
    name=WSDL,
    description={Web Services Description Language \url{https://www.w3.org/TR/wsdl20/}}
}

\newglossaryentry{rest}
{
    name=REST,
    description={Representational state transfer. Architekturstil für Webservices}
}